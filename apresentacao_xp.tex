\documentclass[slidestop,usepdftitle=false]{beamer}
\usepackage[accumulated]{beamerseminar}
\usepackage{beamertexpower}
\usepackage{beamerthemeshadow}

\usepackage[brazil]{babel}
\usepackage[T1]{fontenc}
\usepackage[utf-8]{inputenc}


\title{Extremme Programming}
\author[Jônatas Davi Paganini]{
  \url{http://ideia.me}    |    \textbf{jonatasdp@gmail.com}
}

\begin{document}
\begin{slide}

\maketitle

\newslide

\tableofcontents

\end{slide}

\section{Princípios}
\frame{
  \frametitle{Princípios de uma equipe}
  \begin{slide}
  {
  \begin{description}
    \item[Oportunidade]    \step{ Como é encarada cada situação de um projeto }
    \item[Diversidade]     \step{ Quantos tipos de pessoas contribuem para o projeto? }
    \item[Passos de Bebê]  \step{ Uma coisa de cada vez, em pequenos passos }
    \item[Auto-semelhança] \step{ boas práticas devem ser replicadas }
    \item[Benefício Mútuo] \step{ Programador feliz + Cliente feliz + Gerente feliz }
    \item[Economia]        \step{ O que gera mais retorno financeiro para o cliente? }
    \item[Falha]           \step{ Experimente, descubra, tente, falhe! }
    \item[Fluidez]         \step{ Software não se faz em fábricas }
    \item[Metáfora]        \step{ Você sabe o que é Lixeira e Janela no computador? }
    \item[Humanismo]       \step{ Programador também têm sentimentos } 
    \item[Melhoria]        \step{ Software estraga } 
    \item[Qualidade]       \step{ Quem não gosta? }
    \item[Reflexão]        \step{ Saber por que você está fazendo isso }
    \item[Responsabilidade Aceita] \step{ Tarefas devem ser aceitas ao invés de atribuídas }
  \end{description}
  }
\end{slide}
}

\section{Valores}
\frame{
  \frametitle{Valores mantidos na equipe}
  \begin{slide}
  {
  \begin{description}
    \item[Comunicação]  \step{ Fale, escute, converse }
    \item[Coragem]      \step{ Encare a situação }
    \item[Feedback]     \step{ Receba e dê o seu feedback }
    \item[Respeito]     \step{ Respeite as pessoas com quem trabalha }
    \item[Simplicidade] \step{ Seja simples: não tente complicar }
  \end{description}
  }
\end{slide}
}


\section{Práticas}
\frame{
  \frametitle{Práticas Primárias}
  \begin{slide}
  {
  \begin{description}
    \item[Ambiente Informativo] \step{ Quadro de informações do projeto }
    \item[Build de Dez Minutos] \step{ Build em no máximo 10 minutos }
    \item[Ciclo Semanal] \step{ Entregue um fragmento de software em 1 semana  }
    \item[Ciclo Trimestral] \step{ Entregue uma versão em 3 meses  }
    \item[Desenvolvimento Orientado a Testes] \step{ Escreva testes, depois programe }
    \item[Design Incremental] \step{ Crie o código mínimo para suprir a necessidade }
    \item[Equipe Integral] \step{ O cliente faz parte da equipe }
    \item[Folga] \step{ Um dia por semana para trabalhar em tarefas técnicas }
    \item[Estórias] \step{ Cenários de software }
    \item[Integração Contínua] \step{ Software atualizado e compartilhado constantemente }
    \item[Programação em Par] \step{ Piloto e co-piloto }
    \item[Trabalho Energizado] \step{ Trabalhar de forma inteligente }
  \end{description}
  }
\end{slide}
}

\frame{
  \begin{slide}
  \frametitle{Práticas Corolárias}
  {
  \begin{description}
    \item[Análise da Raiz do Problema] \step{ Detecção de problemas  }
    \item[Base de Código Unificada] \step{ Evite o disperdício de códigos fonte }
    \item[Código Coletivo] \step{ Todos devem conhecer todos os códigos }
    \item[Código e Testes] \step{ Artefatos permanentes no projeto }
    \item[Continuidade da Equipe] \step{ Mantenha boas equipes que trabalham juntas }
    \item[Contrato de Escopo Negociável] \step{ Custo, prazo e escopo não são previsíveis }
    \item[Envolvimento do Cliente Real] \step{ Usuários finais também dão pitacos no projeto }
    \item[Equipes que Encolhem] \step{ A medida que a capacidade de produção aumenta... }
    \item[Implantação Diária] \step{ Claro, se você tiver menos que 5 bugs por ano... }
    \item[Implantação Incremental] \step{ Grande migrações não funcionam  }
    \item[Pagar Por Uso] \step{ Revolucione os objetivos do seu software }
  \end{description}
  }
\end{slide}
}

\frame{
  \frametitle{Outras Práticas}
  \begin{slide}
  {
  \begin{description}
    \item[Reunião em Pé] \step{ Sem embromation }
    \item[Refatoração] \step{ Melhoramento contínuo do código }
    \item[Metáfora] \step{ Aprimore o relacionamento com o cliente }
  \end{description}
  }
\end{slide}
}

\frame{
  \frametitle{Referências e outros recursos}
  \begin{slide}
  {
  \begin{description}
    \item[Site da Improve It - \url{http://improveit.com.br/xp}]
    \item[XP - \url{http://www.extremeprogramming.org}]
    \item[Meu site: \url{http://ideia.me}]
    \item[Apresentação: \url{http://ideia.me/apresentacao\_xp.pdf}]
    \item[Em Latex: \url{http://ideia.me/apresentacao\_xp.tex}]
  \end{description}
  }
\end{slide}
}

\end{document}

